\documentclass[12pt,]{tufte-handout}
\usepackage[mono=false]{libertine}
\usepackage{amssymb,amsmath}
\usepackage{ifxetex,ifluatex}
\usepackage{fixltx2e} % provides \textsubscript
\usepackage{pmboxdraw}
\usepackage[utf8]{inputenc}

\ifnum 0\ifxetex 1\fi\ifluatex 1\fi=0 % if pdftex
  \usepackage[T1]{fontenc}
  %% \usepackage[utf8]{inputenc}
\else % if luatex or xelatex
  \ifxetex
  \else
  \usepackage{fontspec}
  \fi
  \defaultfontfeatures{Ligatures=TeX,Scale=MatchLowercase}

    \setmonofont[Mapping=tex-ansi]{inconsolata}
\fi
% use upquote if available, for straight quotes in verbatim environments
\IfFileExists{upquote.sty}{\usepackage{upquote}}{}
% use microtype if available
\IfFileExists{microtype.sty}{%
\usepackage{microtype}
\UseMicrotypeSet[protrusion]{basicmath} % disable protrusion for tt fonts
}{}
\PassOptionsToPackage{usenames,dvipsnames}{color} % color is loaded by hyperref
\hypersetup{
            pdftitle={Discovering Haskell},
            pdfauthor={FIT2102 Programming Paradigms},
            colorlinks=true,
            linkcolor=RoyalBlue,
            citecolor=Blue,
            urlcolor=RoyalBlue,
            breaklinks=true}
\urlstyle{same}  % don't use monospace font for urls
\usepackage{color}
\usepackage{fancyvrb}
\newcommand{\VerbBar}{|}
\newcommand{\VERB}{\Verb[commandchars=\\\{\}]}
\DefineVerbatimEnvironment{Highlighting}{Verbatim}{commandchars=\\\{\}}
% Add ',fontsize=\small' for more characters per line
\newenvironment{Shaded}{}{}
\newcommand{\KeywordTok}[1]{\textcolor[rgb]{0.00,0.44,0.13}{\textbf{{#1}}}}
\newcommand{\DataTypeTok}[1]{\textcolor[rgb]{0.56,0.13,0.00}{{#1}}}
\newcommand{\DecValTok}[1]{\textcolor[rgb]{0.25,0.63,0.44}{{#1}}}
\newcommand{\BaseNTok}[1]{\textcolor[rgb]{0.25,0.63,0.44}{{#1}}}
\newcommand{\FloatTok}[1]{\textcolor[rgb]{0.25,0.63,0.44}{{#1}}}
\newcommand{\ConstantTok}[1]{\textcolor[rgb]{0.53,0.00,0.00}{{#1}}}
\newcommand{\CharTok}[1]{\textcolor[rgb]{0.25,0.44,0.63}{{#1}}}
\newcommand{\SpecialCharTok}[1]{\textcolor[rgb]{0.25,0.44,0.63}{{#1}}}
\newcommand{\StringTok}[1]{\textcolor[rgb]{0.25,0.44,0.63}{{#1}}}
\newcommand{\VerbatimStringTok}[1]{\textcolor[rgb]{0.25,0.44,0.63}{{#1}}}
\newcommand{\SpecialStringTok}[1]{\textcolor[rgb]{0.73,0.40,0.53}{{#1}}}
\newcommand{\ImportTok}[1]{{#1}}
\newcommand{\CommentTok}[1]{\textcolor[rgb]{0.38,0.63,0.69}{\textit{{#1}}}}
\newcommand{\DocumentationTok}[1]{\textcolor[rgb]{0.73,0.13,0.13}{\textit{{#1}}}}
\newcommand{\AnnotationTok}[1]{\textcolor[rgb]{0.38,0.63,0.69}{\textbf{\textit{{#1}}}}}
\newcommand{\CommentVarTok}[1]{\textcolor[rgb]{0.38,0.63,0.69}{\textbf{\textit{{#1}}}}}
\newcommand{\OtherTok}[1]{\textcolor[rgb]{0.00,0.44,0.13}{{#1}}}
\newcommand{\FunctionTok}[1]{\textcolor[rgb]{0.02,0.16,0.49}{{#1}}}
\newcommand{\VariableTok}[1]{\textcolor[rgb]{0.10,0.09,0.49}{{#1}}}
\newcommand{\ControlFlowTok}[1]{\textcolor[rgb]{0.00,0.44,0.13}{\textbf{{#1}}}}
\newcommand{\OperatorTok}[1]{\textcolor[rgb]{0.40,0.40,0.40}{{#1}}}
\newcommand{\BuiltInTok}[1]{{#1}}
\newcommand{\ExtensionTok}[1]{{#1}}
\newcommand{\PreprocessorTok}[1]{\textcolor[rgb]{0.74,0.48,0.00}{{#1}}}
\newcommand{\AttributeTok}[1]{\textcolor[rgb]{0.49,0.56,0.16}{{#1}}}
\newcommand{\RegionMarkerTok}[1]{{#1}}
\newcommand{\InformationTok}[1]{\textcolor[rgb]{0.38,0.63,0.69}{\textbf{\textit{{#1}}}}}
\newcommand{\WarningTok}[1]{\textcolor[rgb]{0.38,0.63,0.69}{\textbf{\textit{{#1}}}}}
\newcommand{\AlertTok}[1]{\textcolor[rgb]{1.00,0.00,0.00}{\textbf{{#1}}}}
\newcommand{\ErrorTok}[1]{\textcolor[rgb]{1.00,0.00,0.00}{\textbf{{#1}}}}
\newcommand{\NormalTok}[1]{{#1}}
\IfFileExists{parskip.sty}{%
\usepackage{parskip}
}{% else
\setlength{\parindent}{0pt}
\setlength{\parskip}{6pt plus 2pt minus 1pt}
}
\setlength{\emergencystretch}{3em}  % prevent overfull lines
\providecommand{\tightlist}{%
  \setlength{\itemsep}{0pt}\setlength{\parskip}{0pt}}
\setcounter{secnumdepth}{0}
% Redefines (sub)paragraphs to behave more like sections
\ifx\paragraph\undefined\else
\let\oldparagraph\paragraph
\renewcommand{\paragraph}[1]{\oldparagraph{#1}\mbox{}}
\fi
\ifx\subparagraph\undefined\else
\let\oldsubparagraph\subparagraph
\renewcommand{\subparagraph}[1]{\oldsubparagraph{#1}\mbox{}}
\fi

% set default figure placement to htbp
\makeatletter
\def\fps@figure{htbp}
\makeatother


\title{Discovering Haskell}
\author{FIT2102 Programming Paradigms}
\date{S2 2017 Week 6}

\begin{document}
\maketitle

\section{Preamble}\label{preamble}

Welcome to your week 6 tutorial, first in Haskell! The goal of this
session is to learn about the basic syntax of the language.

First of all you will need to install Haskell. \footnote{The easiest way
  to is to get \href{https://www.haskell.org/downloads}{Haskell
  Platform} for your operating system.

  This should come with all the tools you will need to compile, run, and
  test Haskell code.}

Next, you need to actually \emph{set up} the code. To do so, you will
use a tool called \emph{stack} \footnote{Stack comes with the Haskell
  Platform (there's a checkbox to tick on the Haskell Platform
  installer). Optionally, you can download it separately from the
  \href{https://docs.haskellstack.org/en/stable/README/}{stack website}
  and install it for your operating system.} which takes care of all the
nitty gritty details. In a nutshell, stack creates a contained
environment based on the information provided in the file
\texttt{stack.yaml}, this allows you to work on multiple projects with
different, and possibly conflicting, requirements at the same time.

To test stack, from a command prompt in the home directory
(\texttt{cmd.exe} on windows, \texttt{terminal} on mac and linux) run:

\begin{verbatim}
$ stack new haskellPlatformTestProj
$ cd haskellPlatformTestProj
$ stack setup
$ stack build
\end{verbatim}

The first time you run stack setup it will take a minute or so (it is
installing a local version of GHC under \texttt{C:\textbackslash{}SR} or
\texttt{\textasciitilde{}/.stack}) -- later invocations will use the
cached version and will therefore be very quick.

The default stack installation (\texttt{stack\ new}) will create a
number of files in a project directory. The file \texttt{stack.yaml}
contains build requirements, while \texttt{my-project.cabal} contains
the code requirements -- external libraries and the like. The file you
will have to worry about is \texttt{src/Lib.hs} where the logic resides,
\texttt{app/Main.hs} is used to build an executable to run, and the test
will not be used as of now.

\texttt{stack.yaml} contains build requirements, while
\texttt{my-project.cabal} contains the code requirements -- external
libraries and the like. The file you will have to worry about is
\texttt{src/Lib.hs} where the logic resides, \texttt{app/Main.hs} is
used to build an executable to run. The file \texttt{test/Spec.hs}
contains code to run test cases.

For the Week 6 lab class, we have provided you with a \texttt{stack}
project ready to go. The starter code is in \texttt{src/week6}. There
are three files in this directory: \texttt{Pair.hs}, \texttt{List.hs}
and \texttt{BinTree.hs}. Currently the functions in these files are left
undefined. Your task is to complete the functions. So that the tests
pass.

After unzipping the folder, \texttt{cd} into it from the command line
and run \texttt{stack\ setup}. If all goes well, after this step, your
environment will be set up and you can start hacking. To run code within
the environment, use:

\begin{Shaded}
\begin{Highlighting}[]
\NormalTok{$ }\ExtensionTok{stack} \NormalTok{ghci}
\end{Highlighting}
\end{Shaded}

You should see something similar to:

\begin{verbatim}
$ stack ghci
...
Configuring GHCi with the following packages: haskell-tutes
Using main module: 1. Package `haskell-tutes' component exe:
haskell-tutes with ...\week6\app\Main.hs
GHCi, version 8.0.1: http://www.haskell.org/ghc/  :? for help
[1 of 3] Compiling BinTree  
[2 of 3] Compiling List   
[3 of 3] Compiling Pair  
Ok, modules loaded: Pair, List, BinTree.
[4 of 4] Compiling Main    
Ok, modules loaded: Pair, List, BinTree, Main.
Loaded GHCi configuration from ...
*Main BinTree List Pair>
\end{verbatim}

This is the Glasgow Haskell Compiler interactive interpreter, you will
be able to load and run your code directly in there. This interactive
console has tab completion, information on the code loaded, etc.

For example, you should be able to create a Pair from REPL:

\begin{verbatim}
*Main BinTree List Pair> Pair 2 1
Pair 2 1
\end{verbatim}

If not, the tutors are here to help.

In every file of the tutorial, you will see functions with a type
signature and an \texttt{undefined} in the body, you need to replace
this with your code. \footnote{Comments starting with
  \texttt{\textgreater{}\textgreater{}\textgreater{}} are called
  \emph{doctests}, to verify they work run:

  \texttt{\$\ stack\ test}

  If your code is complete, they should all pass, if not it means you
  still have work to do.}

\section{Exercise 1: Pairs}\label{exercise-1-pairs}

A \texttt{Pair} is an element comprising of two \texttt{Int}, that is
two natural numbers.

Your goal is to use \emph{pattern matching} to access the elements in a
pair, or multiple pairs, in order to apply function to them.

\section{Exercise 2: Binary Tree}\label{exercise-2-binary-tree}

A \texttt{BinTree} is a \emph{recursive data structure}, which means
that each element in a binary tree \texttt{is} a binary tree. Each node
in a binary is therefore either a \texttt{Nil}, or empty tree; or a
\texttt{Node}. A \texttt{Node} has three elements:

\begin{enumerate}
\def\labelenumi{\arabic{enumi}.}
\tightlist
\item
  an integer value;
\item
  a left sub-tree; and
\item
  a right sub-tree.
\end{enumerate}

Your task is to write basic functions on a binary tree.

\section{Exercise 3: List}\label{exercise-3-list}

A \texttt{List} is an extended container, implemented as a
\href{https://en.wikibooks.org/wiki/Haskell/More_on_datatypes\#Named_Fields_.28Record_Syntax.29}{record}
\footnote{Record syntax supports pattern matching as well as accessors.},
which holds information about its inner list:

\begin{itemize}
\tightlist
\item
  \texttt{size}: the size of the list;
\item
  \texttt{elems}: the actual list;
\item
  \texttt{low}: the lowest element in the list;
\item
  \texttt{high}: the highest element in the list.
\end{itemize}

Your goal is to implement functions on this data structure while keeping
the members up to date; e.g., if you add an element to the list, you
have to update the size, and potentially the \texttt{low} or
\texttt{high} members.

\end{document}
